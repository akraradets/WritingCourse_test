% https://www.overleaf.com/learn/latex/Learn_LaTeX_in_30_minutes
\documentclass{article}
\begin{document}

\verb!Math! is used for writing a mathematical equation.
There are multiple ways you can insert an equation.

Inline equation $y = ax + b$ will insert the equation in between sentences.

Display equation $$y = ax + b$$ will insert the equation in the new line.

This is not equal to

Display equation 

$y = ax + b$

will insert the equation in the new line.

You can see, using \verb!$...$! with a new line will result in a new paragraph 
while \verb!$$...$$! won't. \\

Now, there are three ways you can do \textbf{inline equation}.

\begin{enumerate}
    \item $y = ax + b$
    \item \begin{math}
        y = ax + b
    \end{math}
    \item \(y = ax + b\)
\end{enumerate}

And there are the ways for \textbf{Display equation}.

1. 
$$ y = ax + b$$
2.
\[ y = ax + b\]
3.
\begin{displaymath}
    y = ax + b
\end{displaymath}
4.
\begin{equation}
    v = ax + b
\end{equation}

Note that \LaTeX is not recommended to use \verb!$$...$$! anymore.


\end{document}