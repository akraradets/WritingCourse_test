% https://www.overleaf.com/learn/latex/Learn_LaTeX_in_30_minutes
\documentclass{article}
% This library is for inserting an image
\usepackage{graphicx}

% This will let the \ref acts like a hyperlink.
% Basically, you can click the text and jump to the reference item.
% In addition, this allowed you to embed a URL.
\usepackage{hyperref}
\begin{document}

% I create a folder `figures'. Then, load `bci-logo.png' into this folder.
% You can use any folder name you like.
\includegraphics{figures/bci-logo.png}

\newpage
% The image is too big to fit on the page
\includegraphics[width=0.75\textwidth]{figures/bci-logo.png}

This form of \verb!\includegraphics! instructs LATEX to set the figure's width to 75\% 
of the text width—whose value is stored in the \verb!\textwidth! command.

\newpage
\begin{figure}[h]
    \centering
    \includegraphics[width=0.75\textwidth]{figures/bci-logo.png}
    \caption{AIT Brainlab Logo}
    \label{fig:brainlab-logo}
\end{figure}

Now, we wrap the figure with the tag \verb!\begin{figure} ... \end{figure}!. 
Inside, we have \verb!\centering! for centering the figure, 
\verb!\caption! to include the caption (yeah sure),
and \verb!\label! for future reference likes

Yo!!, check out figure~\ref{fig:brainlab-logo}.

Labeling begins with \verb!fig:! is a best practice in \LaTeX. 
You don't need to follow this.
We do it for organizing purposes.  

Positioning the figure can be done. Read this \url{https://www.overleaf.com/learn/latex/Positioning_images_and_tables}.

\end{document}
